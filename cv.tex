%!TEX TS-program = xelatex
\documentclass[]{friggeri-cv}
\usepackage{graphicx}
\usepackage{wrapfig}

\begin{document}
\header{Amir}{Khazaie}{}


% In the aside, each new line forces a line break
\begin{aside}	
  \section{Info}
    Born in June 1995
    ~
    Live in Shahid Nejatollahi Dormitory, Beh Afarin St, Valiasr Square, Tehran.
    ~
    \href{mailto:733amir@gmail.com}{733amir@gmail.com}
    09332750385
  \section{Languages}
    Persian (Native),
    English
  \section{Programming}
	C, C++, Python, Java, Kotlin, Go, HTML, CSS, JavaScript, SQL, BASH
  \section{Frameworks and Platforms}
    Qt, CherryPy, Flask, OpenCV, Scikit Learn, LaTeX, Arduino, Android, Docker, OpenGL, Tkinter, PostgreSQL
  \section{Operating Systems}
  OS X, Linux, Android
  \section{Tools}
  Git, Android Studio, PyCharm, IntelliJ, TexStudio, Gimp, Homebrew, Atom, Zsh, Postman, GoLand, Visual Studio Code
\end{aside}

\section{About Me}

%\begin{wrapfigure}{l}{0.25\textwidth} %this figure will be at the right
%	\centering
%	\vspace{-0.5cm}\includegraphics[width=0.2\textwidth]{me.jpg}
%\end{wrapfigure}

Everyone has a hobby, a thing, mine is programming. Nothing is perfect, there is always room to get better, so I accept the challenge to write good programs and maintain them to stay good. I'm a teamman, to achieve something great you need to work with others.

You can find me with \href{https://t.me/KhazaieAmir}{@KhazaieAmir on Telegram}, \href{https://twitter.com/733amir}{@733amir on Twitter}, \href{https://github.com/733amir}{733amir on Github} or give a visit to my LinkedIn page at \href{https://linkedin.com/in/amirkhazaie}{LinkedIn.com/in/AmirKhazaie} link.

\section{education}

\begin{entrylist}
	  \entry
	{2017 - Now}
	{M.Sc. in Artificial Intelligence}
	{Tehran, Iran}
	{Amirkabir University of Technology}
	  \entry
	{2013 - 2017}
	{B.Sc. of Software Engineering}
	{Isfahan, Iran}
	{Isfahan University of Technology}
	\entry
	{2016}
	{Front End Developer}
	{}
	{Free Code Camp}
  \entry
    {2009 - 2013}
    {Mathematics and Physics}
    {Isfahan, Iran}
    {Shahid Ejei High School}
\end{entrylist}

\section{awards}

\begin{entrylist}
	\entry
	{2017}
	{Ranked 113th among over 3500 teams in IEEEXtreme 11.0}
	{}
	{An algorithmic challenge with 24 hours time to solve the problems. Each hour one problem is represented to competitors. Our score was 1457.}
	\entry
	{2017}
	{2nd Place in Isfahan University of Technology, APA Security Startup}
	{}
	{We designed a secure platform for IoT that had 3 layers:
		1- Wireless Sensor Network (Edge of our platform)
		2- Local Server (Collector of data)
		3- SIP Cloud (Data storage and server).
		
		Name of the team and idea was SIP (Secure Internet of Things Platform).}
	\entry
	{2016}
	{Ranked 1st in 4th Iranian ICPC Challenge - Tehran Site}
	{}
	{}
	\entry
	{2016}
	{Ranked 8th in 18th ACM-ICPC Asia Region - Tehran Site}
	{}
	{}
	\entry
	{2016}
	{Ranked 109th among over 2500 teams in IEEEXtreme 10.0}
	{}
	{}
	\entry
	{2015}
	{Ranked 1st in APA Hackathon}
	{}
	{CTF competition in computer security.}
	\entry
	{2015}
	{Ranked 7th in Sharif Java Challenge}
	{}
	{This competition was about writing an AI that wins some kind of a game.}
\end{entrylist}

\pagebreak

\section{projects}

\begin{entrylist}
	\entry
	{2018}
	{Go Jalaali}
	{Bahamta}
	{A project to convert Gregorian and Jalaali calendar systems to each other. The method implemented here is precision of correct conversion over 3000 years.}
	\entry
	{2018}
	{Noter Android Application}
	{}
	{Android note taking application capable of creating text notes and canvas to draw. It supports folders for grouping notes together and full import/export of data (backup).}
	\entry
	{2017}
	{Pixel Android Application}
	{Rahnema College}
	{An Android application for Pixel social media that is based on photo sharing.}
	\entry
	{2017}
	{Solar System Simulation}
	{}
	{Simulation of solar system that can speed up or slow down. You can move around things and watch everything.}
	\entry
	{2017}
	{A5/1 Cipher Simulator}
	{}
	{Contributing for a stream encryption algorithm simulator that you can work with it in browser.}
	\entry
	{2017}
	{Python for You and Me}
	{}
	{A long Jupyter notebook for learning Python.}
	\entry
	{2017}
	{Cafegap Android Application}
	{}
	{Application for managing participant of Cafegap event at Isfahan University of Technology.}
	\entry
	{2017}
	{Cafegap CherryPy Server}
	{}
	{Server side of Cafegap android application.}
	\entry
	{2016}
	{Funsara CherryPy Server}
	{}
	{Project for E-Commerce course. We created a simple video serving system with Android and Web interface.}
	\entry
	{2016}
	{Iranian Calendar Indicator}
	{}
	{A simple indicator (tested for Ubuntu) that will show day of month.}
	\entry
	{2016}
	{Iran's Calendar Events}
	{}
	{Fetching specific day events from \textit{time.ir} as a Python dictionary.}
	\entry
	{2016}
	{Simon (game)}
	{}
	{Simple web based game that challenges your memory.}
	\entry
	{2016}
	{Tic Tac Toe}
	{}
	{Web based Tic Tac Toe with simple and beautiful user interface.}
	\entry
	{2016}
	{Quotes}
	{}
	{Fetch random quotes from Internet and show it in a simple box.}
	\entry
	{2016}
	{Wikipedia Searcher}
	{}
	{A beautiful and animated search bar for Wikipedia.}
	\entry
	{2015}
	{Race Car}
	{}
	{Remotely controlled car with an Android application over WiFi. Arduino Uno and ESP01 modules used in car as processor and communication device.}
	\entry
	{2015}
	{Judger}
	{}
	{A Python script compile and run written codes in C, C++ and Java with specific input and then compare output of code with correct output and report the result.}
\end{entrylist}
\begin{entrylist}
	\entry
	{2014}
	{Neshani}
	{Isfahan University of Technology}
	{A single page website that helps people find links and online services in Isfahan University of Technology.}
\end{entrylist}

\section{experiences}

\begin{entrylist}
	\entry
	{2018}
	{Golang Developer}
	{Bahamta}
	{During my time here, I worked on a Charity and Admin telegram bots written with Go in Clean architecture.}
	\entry
	{2017}
	{Android Developer Intern}
	{Rahnema}
	{6 weeks of Android programming to build a social media based on image sharing feature. At start I know little Android programming, RESTful APIs and Team work, but at the end we wrote the application that was able to communication with server (written by Java Spring) through RESTful APIs.}
	\entry
	{2017}
	{Co-founder, R\&D, Software Developer}
	{Fanoos}
	{Startup was about a secure and flexible Internet of things infrastructure with inspiration of MQTT and CoAP protocols. We design very simple version of it and talk with some companies for fund. One of them accepted us and we work all the summer and our funder abandoned the project in first days and didn't tell us! So without money our team and project failed.}
	\entry
	{2016 - 2017}
	{Teaching Python}
	{Isfahan University of Technology}
	{Teaching more than 100 students about Python and OOP, GUI, Network, Game, Database with Python. For this classes I used my own handbook named Python for You and Me.}
	\entry
	{2017}
	{Organizer of Local ACM 2017}
	{Isfahan University of Technology}
	{The contest held on Feb 16, 2017. I was in the Technical and Scientific team. Configuring DOMjudge and designing problem G was part of my duties.}
	\entry
	{2014 - 2016}
	{Teacher Assistant}
	{Isfahan University of Technology}
	{Introduction to Computer Programming (Fall 2014, Fall 2015, Fall 2016),
		Advanced Programming With C++ (OOP) (Spring 2016),
		Operating Systems (Fall 2016).}
	\entry
	{2015}
	{Organizer of FOSS Congress}
	{Isfahan University of Technology}
	{Planning and executing Free and Open Source Software Congress. Is was about open technologies that was getting popular. Over 150 people participated for 6 workshops on Golang, Angular.js, Laravel, R, Embedded Systems (Raspberry Pi), Linux.}
\end{entrylist}

\pagebreak

\section{publications}

\begin{entrylist}
	\entry
	{2017}
	{Probing, Design and Implementation of an Internet of Things infrastructure using Publish/Subscribe method.}
	{Isfahan University of Technology}
	{This was my Bachelor's project. I explore was IoT means and what protocols and platforms exist out there. Choosing MQTT protocol I tried to design my own platform. Then with Arduino, Raspberry Pi and some sensors and actuators, I implement a simple working example of my designed platform.}
	\entry
	{2017}
	{Remote Connection to IoT Devices}
	{Hypertext}
	{A paper about MQTT, CoAP and ... protocols in Internet of things that enables them to talk to each other.}
	\entry
	{2016}
	{Python for You and Me}
	{}
	{A Jupyter notebook available at my Github account. This handbook is an introduction to Python programming language.}
\end{entrylist}

\end{document}
